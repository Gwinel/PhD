
\chapter{基于硬件事务内存的缓存行哈希表的实现}

%备选标题:基于RTM的锁省略算法
\section{事务内存的基本概念}
事务内存(TM)是一种简化并发编程的并发控制范式,其源自于数据库管理系统中的事务概念。
它的核心思想是将一段代码标记为一条事务。
在数据库管理系统中,事务必须满足ACID性质,即原子性(Atomicity)、一致性(Consistency)、隔离性(Isolation)和持久性(Durability)。
原子性指的是事务中的动作要么全部执行,要么一个都不执行;一致性指的是任何时刻,数据库必须处于一致性状态,即必须满足某些预先设定的条件;隔离性是指一个事务不能看见其它未提交事务所涉及到的内部对象的状态;而持久性则是指一个已提交的事务对数据库系统的改变必须是永久的。

事务内存继承了事务的ACID性质。
事务内存的原子性是指:事务代码区间的内容要么全部执行,要么全部不执行,不存在事务代码停滞在中间的任何一条语句,如果发生意外而中止了事务的执行,则系统状态会回滚到事务开始执行时的状态继续执行;
事务内存的一致性是指是指:事务要么执行完成并使外界看到其造成的系统状态的变化,要么执行失败并使所有相关状态保持不变。如果有多个事务同时运行,那么从这些事务之外的角度来进行观察,它们对系统状态做出的改变始终是一个接着一个发生的,中间不会有任何交叉。
例如,在(对同一个账户)两个独立且并发的存款和取款事务完成之后,账户余额应该是两个操作所产生的累加效果(取钱是对账户加上一个负数)。
事务内存的隔离性是指:并发执行的多个事务代码区间彼此之间是严格隔离的,本事务无法了解其他事务的局部变更结果,所有事务造成系统状态的变化只有在成功提交之后才对外部可见;
事务内存的持久性是指事务代码执行完成并且成功提交之后,对于系统状态的变更是持久有效的,并不会回退到起始状态,直到有新的事务执行的代码改变这个状态。

事务内存具有两个重要的性质,即事务内存要确保对临界区代码片段的执行具有原子性和隔离性。
满足原子性和隔离性的事务内存可以安全的并行执行,可以取代现有的令人头疼的锁机制与原子原语。
基于锁的并发编程是一种悲观的并发模式,它假设获得锁的线程一定会访问共享数据,从而会造成其他线程的阻塞。
而任意两条访问受锁保护的变量的事务能够并发的执行,并且只有在其中一条事务尝试修改共享数据时才会触发回滚机制。

事务内存被认为是最有希望解决多核处理器编程问题的并发编程方案之一。
它最吸引人的特点是程序员只需在本地对共享数据的访问行为进行预估,并让底层系统确保正确的并发执行。
该模型有望提供细粒度锁机制那样的线程扩展性并能避免一般的锁机制中常见的陷阱,比如死锁等问题。

本章主要研究硬件事务内存对构建并发哈希表的作用。

\section{Intel事务同步扩展}
2012年2月,Intel发布了其支持硬件事务内存的事务同步扩展(Transactional Synchronization Extensions, TSX)指令集,并在其2013年6月发布的Haswell微体系结构的微处理器上实现了对TSX的支持。
标志着硬件事务内存的全面商用化。

TSX对使用锁省略方式编写的多线程并发软件具有明显的加速作用。 
TSX提供一组扩展指令集,允许编程人员指定事务同步的代码区域。
有了事务同步,硬件就能动态的决定是否需要将进入临界区的线程串行化执行。
% 这允许处理器利用由于动态的不必要的同步而隐藏起来的并发性。
这个特性有助于提升程序的并发性。
在Intel TSX的最底层,编程人员指定的代码区域(也称为事务区域)可以事务性地执行。
只有确认事务成功提交后,在执行事务区域期间对于系统状态的变更操作才会对其他逻辑核可见。
完成一次事务执行并提交状态变更信息的过程称为一次原子提交。
这样,编程人员只需使用粗粒度的锁编程就能实现细粒度锁方法才能获得的性能。
如果多个线程同时访问受同一个锁保护的临界区而彼此之间又不存在任何数据冲突,那么这些线程就能并发的执行。

Intel TSX提供两种不同的指令前缀:
一种称为硬件锁省略(Hardware Lock Elision,HLE);一种称为限制性事务内存(Restricted Transactional Memory,RTM)。

\subsection{Intel硬件锁省略}
HLE提供与传统指令集兼容的指令集接口供编程人员编写事务化的程序,还提供两条全新的指令前缀XACQUIRE和XRELEASE。

使用HLE时,将XACQUIRE前缀放置在用于获取保护临界区的锁的指令之前。
处理器将XACQUIRE视为省略写相关的锁获取操作的提示。
即使是获取锁之后在该锁对应的临界区内执行写相应的操作,处理器将锁的地址添加到事务区域的读取集(read set)中,不会向该锁发起任何写入请求。
之后逻辑核进入事务执行状态。
如果锁在XACQUIRE前缀指令之前是可用的,那么后面的其他所有的核都将该锁视作是可用的。
因为当前正在事务化执行的逻辑核不会将锁的地址写入到它的写入集(write set)中,也不对其执行外部可见的写操作,其他的逻辑核仍然可以读取该锁而不会造成数据冲突。
这就允许其他的逻辑核也进入临界区,并发的对该临界区进行操作。
处理器将自动检测事务执行期间发生的任何数据冲突,有必要时中止事务操作。

XRELEASE前缀放置在用于释放保护临界区的锁的指令的前面。
这涉及到对锁的写的问题。
如果XRELEASE正在试图将锁的值恢复到XACQUIRE发起锁请求时所操作的同一个锁的值,那么处理器将省略所有与即将释放的锁相关的外部锁请求,并且该锁的地址不会被添加到写入集中。
然后处理器尝试提交事务执行结果。

使用HLE时,有多个线程同时执行到被同一个锁保护的临界区时,如果它们执行的操作不会相互之间产生数据冲突的话,这些线程可以并发执行。
如果无法事务地执行事务代码区域,处理器将不使用省略而以普通的方式执行代码区域。
HLE确保软件有与基于传统锁方法相同的前向执行的能力。
即在不支持HLE的硬件平台上,使用HLE指令前缀的程序也能得到正确的执行结果。
不支持HLE的硬件会忽略掉XACQUIRE和XRELEASE前缀提示,也不会执行任何省略操作。
更为重要的是,HLE很好的与现有的基于锁的编程模型兼容。
在基于锁的编程模型中不正确的使用XACQUIRE和XRELEASE前缀提示不会引起功能错误,但它可能会暴露代码中已经存在的潜在的漏洞。
对于一次成功的HLE执行,锁和临界区代码都必须遵循特定的指导原则~\cite{Intel2015Intel}。

\subsection{Intel限制性事务内存}
RTM为事务的执行提供了更为灵活的接口。
它提供三条新的指令——XBEGIN, XEND和XABORT——供用户启动,提交和中止一次事务执行。

XBEGIN指令用于指定事务代码区域的起始位置,XEND指令表示事务代码区域的结束。
XBEGIN指令携带一个操作数,该操作数用于表示它到回滚指令地址的相对偏移量。
如果RTM代码区域无法成功的事务化执行,将回退到该操作数指向的地址继续执行。

处理器有很多原因中止RTM的事务执行。
硬件自动检测事务中止条件,退回到回滚指令处,从XBEGIN开始的指令时的体系结构状态和EAX寄存器更新的描述事务中止的状态重新开始执行。

XABORT指令允许用户显示的中止RTM代码区域的执行。
它携带一个8位的直接参数,这个参数被加载到EAX寄存器中并将在发生RTM中止后对软件可见。具体参数的意义见表~\ref{tab:rtm_status}~。

\begin{table}[htbp]
  \centering
  \caption{RTM的中止状态定义}
  \label{tab:rtm_status}
  \begin{tabular}{ll}
    \toprule
       EAX寄存器比特位 & 说明 \\
    \midrule
      0     & 由XABORT指令引起的中止 \\
      1     & 该事务有可能在Retry之后成功\\
      2     & 该次中止是因数据冲突而引起的 \\
      3     & 该次中止是因内部的缓存溢出引起的\\
      4     & 该次中止是因调试断点引起的 \\
      5     & 该次中止是因事务嵌套引起的 \\
      23:6  & 保留位\\
      31:24 & XABORT参数(只有位0被置位时有效) \\
    \bottomrule
  \end{tabular}
\end{table}

RTM指令没有任何与它们相关联的数据存储位置。
虽然硬件无法保证RTM区域总是能够成功地提交事务,但大多数遵循Intel技术手册中的指导原则的事务都有可能成功地提交。
使用RTM进行编程时,用户必须设置回退路径,用以确保在事务中止后程序的前向执行。
回退路径的设置可以是传统的锁方法或非阻塞方法,比如,通过简单的锁申请然后以非事务的方式执行指定的代码区域。
此外,在特定的实现中发生中止的事务可能会在接下来的实现中完成事务的执行。
因此,用户必须确保事务性区域代码的路径和回退路径上的代码序列在功能上是完备的。

使用Intel RTM必须获得相应的硬件支持,在不支持RTM的平台上无法完成编译。
这是RTM有别于HLE的地方之一。

\subsection{RTM的Lemming效应}
\textit{Lemming}效应是指正在运行中的事务之间陷入一种相互阻挠对方进入事务执行的状态。
具体地,任意获得锁的线程会对锁变量进行修改,同时强制其他当前正在该锁上执行锁省略过程的线程中止。
这些被中止的线程随后会尝试重新事务的执行或者尝试获取锁。
当存在多个线程都处于这一过程时,这些线程会陷入一种持续阻挠对方回退到回退路径上执行的状态,使得锁省略过程长时间处于停滞状态。
通俗的讲,Lemming效应就是RTM事务在运行过程中由于事务中止而引起的串行开销。

在RTM中,受锁保护的代码段执行时通过启动一条事务并“虚”持有锁,但实际上并没有真正持有该锁。
也就是说,锁被读取,如果它的状态被解锁,那么它就被放置在被锁住的事务读集中,而不影响锁的状态,这仍然保持解锁状态。
然而,当事务中止时(例如发生冲突),它会执行回退然后以非事务性的执行方式获取锁,并将锁写入。
被中止的线程获取的全局可见的锁与RTM事务时试探性加载的锁发生冲突,由于在锁的位置上发生了冲突,所以会导致所有这些锁被中止。
此外,当新的线程到达临界区时发现当前锁已经被其他线程持有时,新的线程不会启动它们的事务。
在公平锁的情况下,获取锁时产生的冲突会使得线程无法并行的执行,这种情况一直会持续到经历一个没有线程试图访问锁的静默期为止。
这种引起不必要的串行化从而限制并发度的现象称为\textit{Lemming}效应~\cite{Dice2008Applications}。


如果是在中止率很低的情况下,\textit{Lemming}效应的影响可以忽略不计。
当中止率升高时,\textit{Lemming}效应的影响会越来越严重。
从本质上说,\textit{Lemming}效应使得事务从中止中恢复过来的成本更高,并且引起不必要的串行化执行。

在事务的执行过程中,常常由于冲突而导致事务中止。
事务申请获取的锁当前被其他线程占有从而被迫中止,但是持有该锁的线程可能很快执行完成并释放锁。
因此,大部分被中止的事务经过一次或者多次重试之后可以成功获取锁,完成提交。
Intel在TSX开发者手册中提倡使用RTM进行锁省略编程时适当的进行RTM重试有助于提升性能。
使用RTM编写锁省略代码时,如下行为会引起\textit{Lemming}效应的加剧:

\textbf{第一,过快的进行RTM重试。}当检测到发生中止时立即进行重试。
考虑这么一种情况,设定最大重试次数为5次(重试5次仍然无法成功提交则放弃事务执行转而使用常规方式申请锁执行),假设现在有t$_1$和t$_2$两个线程在并行的执行事务代码,t$_1$长时间的持有锁。
这时t$_2$尝试申请锁,但发现当前锁不空闲,于是立即开始重试,重试次数很快达到5次,此时t$_1$仍然没有释放锁,t$_2$跳转到回退路径上执行,然后使用标准方式申请锁完成执行。
如果有更多的线程并行的处于上述状态,它们将陷入\textit{Lemming}效应的魔咒。
这些重试达到最大次数进入回退路径执行的线程排队等待锁执行,长时间无法再一次进入到事务执行。
对性能造成损耗。

\textbf{第二,设定的RTM重试次数过大。}一般地,在使用RTM进行锁省略编程时通常能够提升性能。
而且,对于不同的工作负载最大重试次数是弹性变化的。
但是,当设定的重试次数过大时,在处理其他工作负载时会有加剧\textit{Lemming}效应的风险。
同样以t$_1$和t$_2$为例,假设t$_1$和t$_2$都尝试事务执行,它们进入临界区,一定的时间后同时造成对方中止。
然后,它们进行重试,又一次遇到上述情况,如此循环若干次。
摆脱这种纠缠不清的状态的方法是其中一方跳转到回退路径上执行。
但是,如果重试次数过高,它们在进入回退路径之前会浪费较长的时间。

\textbf{第三,不设定回退路径,不停的进行重试。}除了冲突可能造成事务中止之外,还有其他原因(比如执行TSX不兼容的指令,动态链接库,页脏位以及其他异常等)也会造成中止,TSX不能保证每次事务执行都能成功。
不设置回退路径可能导致程序挂起。

\textbf{第四,没有将锁变量放入读集合内。}锁省略依赖于其读取集(read set)中的锁变量,用来确保事务地执行和真实的锁持有者之间的完全同步。
这要求在事务执行过程中至少读取一次锁变量。
如果该锁只使用单一的锁变量,那这个过程将由RTM自动完成,但是当锁有多个锁变量的时候,即便是有RTM的情况下,这个过程也可能会出错。

\textbf{第五,在事务执行完成时使用\textit{xtest}代替锁的状态检测。}

\section{软件辅助的硬件锁省略技术}

Haswell的硬件事务内存使用非常简单的“请求者至上”(requestor wins)的冲突解决策略~\cite{2011Intel}。
具体地是指在事务执行的过程中,如果有一致性消息(读或者写)到达写入集中的缓存行,或者由于写入而导致的驱逐到达读取集中的缓存行,则该事务被中止。
这种策略存在的问题是,它既不能避免线程“饥饿”,又容易引起“活锁”~\cite{Bobba2007Performance}。
Y.Afek通过实验证明事务容易出现伪中止(suprious aborts)现象~\cite{Afek2014Software},这种异常的中止无法用数据冲突或者读/写集的溢出来进行解释。
这种异常的中止表明即便是运行完美的无冲突的工作负载,也可能出现由于Lemming效应而引起的性能损耗。

在使用硬件事务内存设计并发哈希表时,为了减轻
\textit{Lemming}效应对性能造成的负面影响,Intel的开发者手册中建议对被中止的事务进行重试~\cite{Intel2015Intel}。
然而,从上一节中介绍了几种可能加剧\textit{Lemming}效应的原因,过快、过度的使用RTM重试或者不对RTM重试的次数做出限定同样会引起\textit{Lemming}效应。
Y.Afek等人通过实验也表明这种简单的技术方案并不能完全解决这个问题~\cite{Afek2014Software},尤其是当\textit{Lemming}效应很严重时,比如使用公平锁或者高并发度情况下,单纯的依靠RTM重试机制不能有效缓解\textit{Lemming}效应的副作用。
为了有效地抑制\textit{Lemming}效应造成的性能下降问题,他们提出了两种软件辅助方案:软件辅助的锁删除方法和软件辅助的冲突检测方法。

\subsection{软件辅助的锁删除方法}

上一节对RTM的Lemming效应进行介绍时提到,RTM事务携带锁在读取集中运行会引起Lemming效应。
也就是说,尽管RTM事务只是“虚”持锁,但是这个锁仍然会在某个线程中止时制造事务中止的链式反应。
有一种考虑是,如果硬件已经通过一致性协议检测到了冲突,为何不直接避开锁启动事务?
单纯的实现锁的删除会造成两个方面的问题:其一,多个线程会由于数据冲突而彼此中止对方执行的事务,从而造成没有任何线程能够前向执行,线程可能会陷入“活锁”(live-lock)状态;其二,如果有线程持有锁以非事务的方式执行,那么与该线程并发执行的事务线程的事务会观察到不一致的状态(比如由非事务线程写入的中间状态会被事务线程观察到)。
这部分内容将尝试实现锁删除方法,以牺牲不透明度(opacity)为代价(允许事务线程观察到非事务线程写入的中间状态),确保不会出现“活锁”,并避免锁删除带来的两个方面的问题。

解决由于线程饥饿或者活锁引起的线程前向问题,可以通过设置前向保证机制——在事务线程提交时,线程回退,并使用锁完成提交。
但是,简单的让被中止的线程通过获取锁避免饥饿的做法会导致不正确的执行结果。
这是由于运行在临界区内的线程的非原子性造成的。
由这样的线程执行的内存更新操作是全局可见的,使得并发执行的事务线程可以读取到不一致的状态。
下面结合一个具体实例予以说明。

\textbf{错误示例:}图~\ref{fig:error_example}中的代码片段$C_1$和$C_2$受同一个锁L保护。
假设线程$t_1$不访问锁L,事务地执行$C_1$,并且读取到X的值为0。
此时,另一个线程$t_2$,非事务性地执行代码$C_2$,申请锁L,并将$Y = 1$写入。
紧接着,线程$t_1$从内存中读取变量Y。
由于变量Y不在$t_1$的读集中,所以线程$t_2$对变量Y的写入操作与线程$t_1$读取变量Y的操作不存在数据冲突,此时$t_1$读取到的变量Y的值为1。
随后$t_1$执行完毕并提交。
因此,$t_1$观察到不一致的状态:$X = 0$且$Y = 1$。
(在上面的例子中,如果$t_2$事务性地运行,$t_1$要么能够读取到$t_2$的全部写入结果,要么全部读取不到,不存在只读取到其中的某一个)。

\begin{figure}[htbp]
\centering
\includegraphics[width=0.7\textwidth]{error_example}
\caption{错误示例}\label{fig:error_example}
\end{figure}

\SetKwProg{Fn}{Function}{}{}
\begin{algorithm}[htbp]
\SetAlgoLined
\Fn{lock()}{
  $retries = $ 0 ~~\\
  \textbf{speculative path:}~~\\
  XBEGIN (line 6)   \tcc*[f]{如果发生中止,跳转到第6行重试}\\ 
  \Return
  ~~\\
  \textbf{fallback path:}~~\\
  $retries = retries + $ 1 ~~\\
  \eIf{$retries < MAX\_RETRIES $}{
    goto Line 3
  }{
    lock.lock() \tcc*[f]{执行标准锁请求}
  }
}
\caption{SLR的加锁方法}
\label{algo:slr_lock}
\end{algorithm}

为了消除Lemming效应对硬件事务内存的影响,引入第一种软件辅助技术——软件辅助的锁删除技术,Software-assisted lock removal(SLR)。
SLR可以视为对Rajwar和Goodman\cite{rajwar2002transactional}硬件锁删除技术的软硬件混合实现。
Rajwar和Goodman\cite{rajwar2002transactional}观察到,只要事务内存能够为冲突的事务提供前向保证,那么就可以不需要加锁操作就能执行具有相同临界区的事务(也就是通过启动事务代替申请锁,提交事务代替释放锁)。
但是这种方法需要依赖于硬件冲突管理策略能够保证线程无饥饿,从而避免出现活锁问题(存在数据冲突的事务陷入彼此中止对方的死循环)。
而HasWell的事务内存采用的“请求者至上”的冲突管理策略既不能保证线程无饥饿,又不能避免活锁。
SLR使用如下方法解决线程活锁问题:它使用硬件事务内存事务地执行临界区,在事务执行过程中不需要访问锁,直到准备提交时才会读取锁。
提交时,如果当前锁没有被其他线程持有,则获取该锁并完成提交;否则,该事务被中止并进行重试(retries)。
如果重试了若干次之后仍然未能成功提交,则放弃事务地执行,改用非事务的方式通过直接获取锁完成本次操作。
在使用SLR时,线程获取锁的行为既不会与正在运行中的事务产生冲突,也不会阻止其它到达的线程推测性地(speculatively)启动其事务。
但是这个方法有一个缺点,它会损失一部分不透明性(opacity),即推测性事务可能会与持有锁的事务并行的运行,此时推测性事务可能会观察到不一致的状态(这会导致推测性事务提交失败并中止)。
R. Guerraoui\cite{guerraoui2008correctness}的研究表明,由于事务的沙箱(sand boxing)机制,这种不透明性在大多数情况下是安全的。

SLR的加锁和解锁过程如算法~\ref{algo:slr_lock}~和~\ref{algo:slr_unlock}~所示。
在算法~\ref{algo:slr_lock}~中,第3行为事务代码区域的起始位置。
如果在事务执行时发生中止,则跳转到第6行的回退路径上执行,在回退路径内,首先对中止的事务进行重试,如果重试的次数超过预先限定的门限值MAX\_RETRIES,则放弃事务执行,转而用普通的方式通过申请锁完成本次执行(第10行)。

\SetKwProg{Fn}{Function}{}{}
\begin{algorithm}[htbp]
\SetAlgoLined
\Fn{unlock()}{
   \eIf{$ XTEST() $}{ \tcc*[f]{如果本次为事务执行,返回TURE}~~\\
      \eIf{lock is $ TRUE $}{
          XABORT       \tcc*[f]{中止事务执行}
      }{
          XEND
      }
   }{
      lock.unlock()   \tcc*[f]{以标准解锁方式解锁}
   }
}
\caption{SLR的解锁方法}
\label{algo:slr_unlock}
\end{algorithm}


\subsection{软件辅助的冲突检测方法}

算法~\ref{algo:scm_lock}~和~\ref{algo:scm_unlock}~描述了SCM的加锁和解锁的过程。

\SetKwProg{Fn}{Function}{}{}
\begin{algorithm}[htbp]
\SetAlgoLined
\Fn{lock()}{
  $retries \gets 0$ \; 
  \textbf{primary path:}~~\\
  XBEGIN (line 6)   \tcc*[f]{如果发生中止,跳转到第6行重试} ~~\\
  条件成熟时调用HLE或者SLR的$lock()$方法\;
  \Return
  ~~\\

  \textbf{serializing path:}~~\\
  \eIf{aux\_lock\_owner  is $ FALSE $}{
      $ retries \gets retries + $1\;
  }{
      aux\_lock.lock()      \tcc*[f]{辅助锁申请锁}~~\\
      $aux\_lock\_owner \gets TURE$
  }
  \eIf{ $retries < MAX\_RETRIES $}{
      goto Line 3
  }{
      main\_lock.lock()
  }
  }
\caption{SCM的加锁方法}
\label{algo:scm_lock}
\end{algorithm}

软件辅助的冲突管理技术(Software-assisted Conflict Management, SCM)。
SLR方法虽然能够很好的抑制\textit{Lemming}效应,但是这个方法有一个缺点,它会损失一部分不透明性(opacity)。
SCM是另一种抑制\textit{Lemming}效应的软件辅助方法,它通过使用简单的冲突管理技术,允许不冲突的线程继续运行其试探性地基于RTM的事务,而不受产生冲突的线程的影响。
为此,在锁的实现中添加一个串行化的路径,其中一个被中止的线程必须获得一个独立的辅助锁(不使用锁省略)以便重新加入与其他线程的推测执行。使用这种方法,发生冲突的线程之间被串行化,而不影响其他线程的并行执行。
如果线程在多次冲突中失败,那么它必须放弃推测执行转而申请原始锁。

SCM方法使用了两种锁,一种为主锁(\textit{main\_lock}),一种为辅助锁(\textit{aux\_lock})。
主锁使用的是基于RTM/SLR机制中的一种,而辅助锁是一种使用传统的锁方法实现的。
只有在以标准的非事务执行方式执行代码区域时才会申请使用辅助锁(算法~\ref{algo:scm_lock}~第9、10行)。
辅助锁会将所有陷入冲突的线程召集到一起,然后将顺序地执行。
当某一事务被中止时,被中止的线程以非事务性的方式申请辅助锁,随后重新加入到对原始临界区的推测执行队列中。
这种为了重新加入到推测执行队列而申请辅助锁的过程称为串行化的路径(serializing path)。
线程在进入串行化路径之前会对事务进行重试。

\SetKwProg{Fn}{Function}{}{}
\begin{algorithm}[htbp]
\Fn{unlock()}{
\eIf{XTEST()}{
  调用HLE或者SLR的$unlock()$方法
  XEND
}{
  main\_lock.unlock()
}
\If{$ aux\_lock\_owner = TRUE $}{
  aux\_lock.unlock() ~~\\
  $aux\_lock\_owner \gets FALSE$
}
}
\caption{SCM的解锁方法}
\label{algo:scm_unlock}
\end{algorithm}
使用SLR方法时,当线程放弃事务执行转而以非事务的方式申请锁时会引起的程序的前向问题,结合使用SCM可以解决这一问题。

上述的两种软件辅助方法已经使用标准的pthread锁接口封装到动态库中。
可以直接在程序中使用上述方法而无需对代码进行调整,也无需重新编译。
\section{基于HTM的并发哈希表的设计}

\subsection{问题引入}

%哈希表操作示意图
\begin{figure}[htbp]
\centering
\includegraphics[width=0.9\textwidth]{hashtable}
\caption{串行执行五个操作的哈希表}\label{fig:hashtable}
\end{figure}

哈希表用于在线性时间内对键值对建立映射关系。
插入和查询是哈希表的两种核心操作,当然哈希表的扩容和删除操作也很重要,但是这里为了简要的引出所要解决的问题,暂时不考虑哈希表的扩容和删除操作。

考虑这么一种场景,有一个简单的如图~\ref{fig:hashtable}所示哈希表,它顺序地执行五次操作,分别是插入、插入、插入、查询、插入。
现在变更需求,希望将这个哈希表设计成并发的,会有怎样的问题呢?

设计高并发的哈希表并不是一件简单的事。
目前有很多方法,比如基于锁的方法,使用原子指令实现的无锁化编程等都可以用于构建高并发度的哈希表。
但是这些方法无疑都会涉及到更加复杂的数据结构的设计,同时也增加了程序的复杂性。

使用粗粒度锁方法(使用少量的锁)实现并发是最简单的方法。
最直观的是使用全局锁,即将整个哈希表作为临界区,只用一个锁进行同步控制。
在这个方法中,每一次对哈希表的操作都是先申请锁,然后执行相应操作,再释放锁。
当锁被某个线程持有时,其他线程无法获得锁,因此也不允许对哈希表进行其他操作。
显然,这种方法的缺陷是致命的,它的效率非常低。

与粗粒度锁对应的方法是细粒度的锁方法,即缩短临界区的长度,增加锁的数量。
具体的方法是将哈希表分成更小的区间,每个区间用一个锁保护,一般的做法是对每个哈希桶设置一个锁。
使用这种方法的好处是,允许多个线程在不存在数据竞争的前提下并发的对哈希表进行操作。
相比于之前采用的整个哈希表用一把锁的情形,显然细粒度锁方法的效率要更高。
但是细粒度锁方法同样也存在缺陷,它引入了不必要的延迟,有更高的内存消耗,并且使得数据结构的设计更加复杂,且需要花费大量精力进行正确性验证。

粗粒度锁方法易用,便于理解,易于调试,唯一的缺陷是在多线程环境下对性能造成阻碍。
在多处理器上这个缺陷是致命的,不可调和的。
细粒度锁方法能够较好的发挥多核处理器的性能,但是细粒度锁方法的实现颇为复杂。
那是否有一种方法既具备粗粒度锁方法简单易用,便于理解,易于调试的特点,又能够保证获得使用细粒度锁方法那样的性能呢?
Intel TSX实现了细粒度锁或者无锁化编程的性能与粗粒度锁方法的简单易用等特点的结合,对于并发数据结构的设计具有重要意义。

图~\ref{fig:hashtable}~中的五个操作,只有两个插入操作存在冲突。
其他的三个操作都是互不干扰的。
因此,使用事务内存可以实现硬件锁省略。
换句话说,使用这种方法获得的性能与执行没有加锁和解锁过程的代码获得的性能非常接近。
造成性能相近的关键原因是这些操作受到硬件事务内存的保护,该硬件取代了锁的功能,实现对临界区的保护。

被映射到同一地址的两个操作由于存在冲突,所以需要交错执行。
但是,在多线程环境中,两个操作有可能被两个线程同时执行。
Intel TSX将确认在这种情况下确实需要加锁以保证代码的正确执行,同时也会因为加锁操作而引入额外的开销。
实际上,出现这种情形时,相互冲突的任务将执行到事务代码里面,一直到处理器检测到这个冲突。
这时,双方都会中止事务代码的执行。
最常用的解决这个问题的方法是让每个任务继续以非事务的方式(常规的获取锁,释放锁的方式)执行。
这就意味着,总有一个任务先获得锁而进入临界区完成操作,而另一个任务则延迟直到先进入临界区的任务执行完毕。

\subsection{缓存行哈希原型描述}

\SetKwProg{Fn}{Function}{}{}
\begin{algorithm}[htbp]
\SetAlgoLined
\SetKwRepeat{Do}{do}{while}%
\SetKwFunction{Insert}{Insert}%
\Fn{Insert(hashtable,key,val)}{
  初始化\;
  $lock \gets \&bucket\rightarrow lock$\;
  $empty \gets NULL$\;
  $empty\_v \gets NULL$\;
  LOCK\_ACQ(lock)\;

  \While{True}{
  \For{$j=0$ \KwTo $ENTRIES\_PER\_BUCKET - 1$}{
    \If{$\&bucket\rightarrow key[j] == key $}{
      LOCK\_RLS(lock)\;
      \KwRet {$false$}\;
    }
    \ElseIf{$empty == NULL $ and $bucket.key[j] == 0$}{
      $empty \gets \&bucket\rightarrow key[j]$\;
      $empty\_v \gets \&bucket\rightarrow val[j]$\;
    }
  }

   \If{$bucket\rightarrow next == NULL$}{
      \eIf{$empty == NULL$}{
       $bucket\rightarrow next \gets clht\_bucket\_create()$\;
       $bucket\rightarrow next \rightarrow key[0] \gets key$\;
       $bucket\rightarrow next \rightarrow val[0] \gets val$\;
      }{
       $\*empty\_v \gets val$\;
       $\*empty \gets key$\;
     }

     LOCK\_RLS(lock)\;
     \KwRet{$true$}\;
  }
  $bucket \gets bucket\rightarrow next$\;
  }
}
\caption{CLHT-lb的插入方法}
\label{algo:clht-lb-insert}
\end{algorithm}

根据第三章中对现有并发哈希表的全面评估与分析的结果,发现缓存行哈希表在多个平台上无论是线程扩展性、性能或者延迟这些方面都具有突出的表现。
因此,在设计基于硬件事务内存的并发哈希表时,参考了缓存行哈希表的数据结构的设计思路。

在前一章中详细介绍了缓存行哈希表(第\ref{sec:clht}节)的设计思想、原理以及使用的同步方法。
这里只对与本章内容密切相关的问题做一个简单的回顾。
缓存哈希表的哈希桶被设计成具有与缓存行相同的大小(64字节),这样设计的好处是在发生缓存未命中时,最多只需要一次缓存行切换操作就能匹配到正确的数据。
这样可以避免由于过多的缓存行切换次数对系统性能造成的损耗。
每一个哈希桶中包含了一个8字节的同步控制字段,用于存放锁(用于实现基于锁的版本)或原子快照(用于实现无锁版本的同步控制)。
基于锁的缓存行哈希表采用细粒度锁(每个哈希桶都设置有一个锁字段)完成对读者线程和写者线程的同步控制。
查询操作遍历键/值对,如果匹配,则返回值。
算法~\ref{algo:clht-lb-insert}~描述了基于锁的缓存行哈希表的插入操作的过程。
更新操作(插入或者删除元素)首先需要执行一次查询以确定该操作可以继续执行(如果插入元素时发现桶内已存在相同元素或者删除元素时发现桶内没有该元素则不进行下面的操作),如果可以继续执行,则持有该桶的锁,直到完成相应的更新操作,完成后释放锁。
如果当前映射到的哈希桶内已经没有足够的空间插入新的元素,则会选择使用指针字段链入一个新的哈希桶,或者触发哈希表扩张操作(resize)。

通过评估的结果还发现缓存行哈希表对NUMA架构的多核系统友好,能够很好的克服访问远程内存节点时过高的延迟开销的问题。

但是,缓存行哈希表同样存在一些不足之处:

第一,较之Cuckoo、Hopscotch等哈希表,缓存行哈希表需要消耗更多的内存。这主要体现在两个方面:
\begin{itemize}
\item 基于链表的实现方法占用更多的内存空间。使用链表组织哈希桶的结构可以保证查询速度以及哈希表的空间利用率,但是每个桶都需要有8字节用于存放指针;
\item 以缓存行大小为粒度的锁实现消耗更多内存。临界区的长度等于一个缓存行的大小,每个哈希桶都包含8字节的信息用于并发控制(锁版本该位置存放锁,无锁版本该位置存放的是原子快照)。
\end{itemize}

第二,细粒度锁方法虽然能够保证线程扩展性以及稳定的性能,但其实现复杂度过高并且难以进行正确性验证。

硬件事务内存为多线程应用的同步提供硬件指令支持,使用硬件事务内存既可以实现无锁化编程,又能用于实现锁。
它以粗粒度的锁实现达到或者接近使用细粒度锁方法所获得的性能。
本文参考缓存行哈希表的设计思想,使用Intel RTM实现了基于硬件事务内存的并发哈希表。
并运用SLR和SCM两种技术用于优化RTM实现的锁以获取更高的性能。

\subsection{基于RTM的锁方法描述}
\label{sec:rtm_lock}

在介绍实现基于RTM的缓存行哈希表所用的同步方法前,先对MCS锁的基本原理进行介绍,并就采用MCS锁的原因予以说明。

% \textcolor{red}{todo:TTAS锁的原理和优缺点!}

实现基于硬件事务内存的缓存行哈希表使用的是MCS自旋锁\cite{mellor1991algorithms},MCS锁的名字来源于其发明人——John Mellor-Crummey和Michael Scott名字的首字母,这个名字并不涉及锁本身的具体意义。
MCS锁是一个链表形式的锁,每一个线程是链表中的一个节点,MCS锁用一个tail指针维护链表中的最后一个节点,每一个节点用一个布尔值$locked$表示自己是否被锁定,以及一个next指针表示在链表中的下一个节点。
\begin{itemize}
\item \textbf{加锁:}线程获取锁时,用SWAP操作将tail指针更新为指向自己;如果tail之前的值为NULL,则表示没有线程在等待,该线程可以直接获得锁进入临界区。
如果tail的值不为NULL,说明前面有线程在等待,那么该线程插入到链表尾端,并将自己的locked标志设置为true,然后在locked上自旋,等待其它线程将locked设置为false,然后退出循环结束等待,进入临界区;
\item \textbf{解锁:}线程A释放锁时,首先判断自己的next节点是否为NULL,如果为NULL,则说明后面没有线程在等待,那么A就是链表中的最后一个线程,随后执行CAS操作,将tail指针指向A,将tail指针设置为NULL并返回,如果CAS执行成功了,表示成功释放了锁,可以直接返回;如果CAS操作失败,说明在执行CAS操作之前有其它线程B先一步插入到链表的尾端并修改了tail指针,B会将A的next指针指向自己,则A需要等待next指针不为NULL。
随后A将B的locked标志修改为false,B获取锁,进入临界区。
\end{itemize}

MCS锁的加锁和释放锁的过程如图~\ref{fig:mcs}所示。
图~\ref{fig:mcs}(a)描述的是多个线程等待获取锁的情形,一共有A、B、C三个线程,其中线程A的locked标志为false,表明线程A正在临界区内执行,B和C先后试图获取锁,由于此时A正在临界区内,所以B和C依次插入到链表内,它们的locked标志都为true。tail指向当前链表内的最末端的节点,图中对应节点C。
图~\ref{fig:mcs}(b)描述的是线程A释放锁的过程,A释放锁时,A将它的next指针指向的节点B的locked标志设置为false,B退出结束自旋,进入临界区。
A将自己从链表中脱离。

选择MCS锁的原因:其一,基于test-and-set(TAS)的自旋锁在多核系统上的扩展性较差。每一次TAS操作都会写入高速缓存,导致所有线程的对应的缓存行都失效,当线程数量较多时,就会产生大量的缓存一致性流量。
MCS锁的优点在于每一个线程只在自己的locked变量上自旋,因此自旋完全可以在对应的核心的L1缓存中完成,不会引起缓存一致性流量和NUMA系统上的跨节点流量,也不会产生内存访问。
只有在释放锁时需要对next指向的线程的locked变量进行一次写入操作。
另外,MCS锁采用的是链表,是一种公平锁,线程按照进入链表的先后顺序排队使用锁。


\begin{figure}[htbp]
\centering
\subfigure[加锁]{\includegraphics[width=0.45\textwidth]{mcs_lock}}
\subfigure[解锁]{\includegraphics[width=0.45\textwidth]{mcs_unlock}}
\caption{MCS锁的工作原理}
\label{fig:mcs}
\end{figure}


\textbf{加锁过程:}
% \textcolor{red}{完善对于加锁过程的描述}
% 实现基于硬件事务内存的缓存行哈希表所用的锁是MCS锁。\\todo mcs的原理和优点!!
算法~\ref{algo:slr-scm-lock}描述的是结合了SLR和SCM两种软件辅助方法的MCS锁。
该算法中,进行RTM重试的次数设定为10,用$MAX\_RETRIES$表示。
首先将$lock$设置成MCS锁(第2行),并将变量$reason$初始化为0(第3行);
算法设计了两条执行路径,一条为speculative路径(4-6行),用于事务执行;一条为fallback路径(7-32行),用于触发了事务中止时执行。
在speculative路径内,指令前缀XBEGIN表示事务代码的起始位置,它携带两个参数:fallback\_path,指明回退路径的地址,reason用于记录当前事务被中止的原因。
如果事务顺利执行并且成功提交,则通过XEND指令前缀结束该事务;
如果由于某种原因触发事务中止时,线程携带触发中止的原因跳转到fallback路径上执行。
进入fallback路径后(第7行),首先将变量$retries$的值加1(第8行),再将$retries$与$MAX\_RETRIES$进行比较,如果进行RTM重试的次数还没有达到上限,则跳转到第4行的speculative路径上,重新尝试运行事务(第10行)。
如果进行重试的次数已经超过了上限值,并且此时线程持有的是辅助锁,则将线程添加到辅助锁队列内(12-14行);否则就以标准的锁申请方式申请获得辅助锁(16-25行),并使线程进入忙等待状态(第21行)。
如果事务被中止的原因是TXN\_MAY\_SUCCEED,进入辅助锁队列内的线程还会进行一轮重试(26-29行)。
最后,设置主锁为"True",以标准方式申请持有主锁。


\SetKwProg{Fn}{Function}{}{}
\begin{algorithm}[htbp]
\SetAlgoLined
\Fn{lock\_mutex\_lock(*mutex)}{
  $*lock \gets  *mutex$\;
  $reason \gets 0$\;
  speculative\_path:~~\\
  XBEGIN(fallback\_path, reason)\;
  \KwRet {$0$}\;

  fallback\_path:~~\\
  $retries \gets retries + 1$\;
  \If{$retries < MAX\_RETRIES$}{
    goto speculative\_path\;
  }
  $*prev \gets NULL$\;
 
  \eIf{$thread\_handle == lock\rightarrow aux\_lock\_owner$}{
    $lock \rightarrow aux\_retries++$
  }{\tcc{以标准方式申请辅助锁}
    $my\_aux\_node.locked \gets true$\;
    $prev \gets \_\_sync\_lock\_test\_and\_set(\&lock\rightarrow aux\_lock, \&my\_aux\_node)$\;
    \If{$prev != NULL$}{
      $prev\rightarrow next \gets \&my\_aux\_node$\;
      \While{$my\_aux\_node.locked$}{cpu\_relax()\;}

      $lock \rightarrow aux\_lock\_owner \gets thread\_handle$\;
      $lock \rightarrow aux\_retries \gets 1$\;
    }
    \If{$reason \& TXN\_MAY\_SUCCEED$ is $true$}{
      \If{$lock\rightarrow aux\_retries \lq MAX\_RETRIES$}{
        goto speculative\_path\;
      }
    }
    \tcc{以标准方式申请锁,这个过程与申请辅助锁一样,省略描述}~~\\
  }
  \KwRet {$0$}\;
}
\caption{结合SLR和SCM的锁方法}
\label{algo:slr-scm-lock}
\end{algorithm}


\textbf{解锁过程:}
算法\ref{algo:slr-scm-unlock}描述了与算法\ref{algo:slr-scm-lock}对应的解锁过程。
在执行解锁操作时,首先使用XTEST指令判断当前执行的内容是否属于事务执行,如果为事务执行(第3行),则进一步判断当前执行的事务代码内是否存在锁嵌套(第4行),如果当前恰好有事务获取了锁正在执行提交操作,则显式的调用XABORT指令中止该事务(第5行),这样可以避免因锁嵌套引起的lemming效应;然后使用XEND指令终止当前正在执行的事务代码。
如果当前线程正处于辅助锁队列内(第8行),则需要使对应的辅助锁标志(第9行)无效,并将辅助锁的重试次数置为0(第10行),随后将辅助锁释放(11-18行)。
如果当前线程不是事务线程(即它以非事务的方式对临界区进行操作),则表明该线程持有的是主锁,则按照标准的锁释放流程释放锁,将锁的状态置为"False"(22-31行)。
最后,将变量$retries$置为0。


\SetKwProg{Fn}{Function}{}{}
\begin{algorithm}[htbp]
\SetAlgoLined
\Fn{lock\_mutex\_unlock(*mutex)}{
  $ *lock \gets *mutex$\;
  % $ *last \gets NULL$\;

  \eIf{$XTEST()$ is $true$}{
    \If{$lock \rightarrow lock$ != $0$}{
      XABORT($1$)\;
    }
    XEND()\;

    \If{$thread\_handle == lock\rightarrow aux\_lock\_owner$}{
    $lock\rightarrow aux\_lock\_owner \gets INVALID\_THREAD\_HANDLER$\;
    $lock\rightarrow aux\_retries \gets 0$\;

    \If{$my\_aux\_node.next == NULL$}{
      \If{\_\_sync\_bool\_compare\_and\_swap(\&val,\&val,NULL) is $true$}{
        \KwRet{$0$}\;
      }
      \While{$my\_aux\_node.next == NULL$}{cpu\_relax()\;}
    }
    $my\_aux\_node.next \gets NULL$\;
    $last\rightarrow locked \gets false$\;
    }
    }
  {
    \tcc{使用标准的方式进行解锁}\
    \If{$my\_node.next == NULL$}{
      \If{\_\_sync\_bool\_compare\_and\_swap(\&val,\&val,NULL) is $true$}{goto unlock\_aux\_lock\;}
      \tcc{接下来重复第15至17行while循环}
    }
    $my\_aux\_node.next \gets NULL$\;
    $last\rightarrow locked \gets false$\;

    unlock\_main\_lock:
    \tcc*[f]{这里重复第9至20的过程}\\
  }
  $retries \gets 0$\;
  \KwRet{$0$}\;
}
\caption{结合SLR和SCM的解锁方法}
\label{algo:slr-scm-unlock}
\end{algorithm}

\subsection{基于HTM的缓存行哈希表}

原始的缓存行哈希表的实现有基于锁和无锁两个版本。
本文设计基于HTM的缓存行哈希表的灵感主要来自其基于锁的版本。
缓存行哈希表采用的是细粒度锁方法,这种方法的好处是使临界区足够短,能有效的减少数据冲突。
但是这种方法实现较为复杂,而且锁的数量过多,维护锁的开销较高。
考虑到硬件事务内存具有简化多线程同步的特点,本文利用算法\ref{algo:slr-scm-lock}中描述的MCS锁实现了基于HTM的缓存行哈希表。
当线程执行对哈希表内元素的操作时,线程首先“虚”持该锁(线程表面上持有锁,但实际上该锁并没有真正意义上被占有,如果有其它线程完成了事务需要进行提交时,仍然可以优先持有该锁),然后事务的执行相应的操作,待执行完毕准备提交时才真正意义上的申请锁,如果当前锁为空闲状态,则占有锁,完成提交;如果提交时锁被其他线程占用,则中止提交,然后进行重试。

基于HTM的缓存行哈希表使用的是全局锁,它将整个哈希表作为临界区的方法本质上是一种粗粒度锁实现,但是通过与原始的细粒度锁实现的缓存行哈希表比较发现,无论是线程扩展性还是吞吐量都具有极强的竞争力,甚至获得了优于使用传统细粒度锁方法实现的缓存行哈希表的性能。

为了进一步探究RTM对细粒度并发哈希表性能的影响,实现了基于RTM的细粒度锁版本的缓存行哈希表。
锁的粒度与原始实现一致,唯一不同的是用基于RTM的MCS锁代替原始实现使用的锁方法,具体的性能表现在第~\ref{sec:rtm-eva}一节的性能评估中展示。


\section{性能评价}
\label{sec:rtm-eva}
\subsection{测试平台和参数设置}
进行实验测试的平台为Linux工作站。
该工作站配备有两个Intel Xeon Broadwell EP/EN/EX处理器,总共有32个物理核(64个逻辑核),内存总容量为64 GB。
CPU的时钟频率为2.1GHz,三级缓存的容量分别为64 KB,256 KB以及40 MB。
该工作站安装的是Ubuntu 16.04 LTS操作系统。

本次实验的源代码使用GCC-4.8.0编译生成可执行文件。
下文中如非特别说明,所有的可执行文件的编译都是使用默认的线程绑定方案(见~\ref{sec:thread_pinning}~),即由操作系统完成线程与核之间的映射。
为了简便起见,实验中所用的键/值对的大小均为64位,所有的查找、插入以及删除请求都是按照预先设定的分布方式通过伪随机数方法生成。
为了便于描述,创建的线程数量用参数\textit{n}表示。所有被创建的\textit{n}个线程都执行相同的工作负载,该工作负载中包含的更新操作请求的比重用\textit{u}\%表示,则查询操作所占的比重为100-~\textit{u}\%(实验中的默认更新比重为10\%)。
更新操作的一半为插入操作,剩下的一半为删除操作。
实验中用到的其它参数如下:\textit{d}表示一次测试运行的时间,单位为毫秒。
\textit{i}表示预先填充至哈希表中的元素的个数,\textit{r}表示键的范围,r的范围为0到2\textit{i}。

\subsection{基于HTM的粗粒度锁实现与细粒度锁CLHT-lb的比较}
本次实验中,设定了大小为1000和1百万两个不同的初始化值,设定为1000的目的是使工作集的大小恰好可以被每个核的私有缓存所容纳,而设定为1百万的目的是使得所运行的工作集的规模超过最后一级缓存的大小,以便测试主存与缓存之间进行数据交换的性能。
每个工作集中更新操作所占比重为10\%,每次测试运行时间为5000毫秒。
实验的最终结果取五次运行结果的平均值,最终得到的性能曲线如图~\ref{fig:htm_coarse_grained}~所示。fine-grained代表基于传统的细粒度锁实现的CLHT-lb版本;slr-scm-mcs代表结合了SLR和SCM方法优化的基于HTM的缓存行哈希表实现。
在初始化元素个数设定为1百万时,两个版本都展现良好的线程扩展性,吞吐量随着参与运算的核的数量的增加而增加,其中slr-scm-mcs性能要略好于fine-grained(约高出20\%左右)。
然而,当初始化元素的规模小于私有缓存的容量时,fine-grained的性能要比slr-scm-mcs的性能要好。
出现这种现象的原因在于:在这个级别的数据集下,缓存行哈希表会遭遇严重的数据冲突,而数据冲突的加剧会造成事务执行频繁的进行重试,并且触发事务中止的概率也更高,事务一旦中止,就必须跳转到回退路径处继续执行。
而从实现的复杂度的角度来分析,基于HTM的缓存行哈希表的实现方式要比传统的细粒度锁方式要更加简单。
在构建并发数据结构时,它只需要使用一个全局锁对临界区进行保护,极大的减少了锁的数量,并且完美的摆脱了使用全局锁会抑制多线程性能的问题。
另外,使用细粒度锁方法需要的内存空间要多于使用HTM时所需的内存空间。
% CLHT的哈希桶的大小设计成等于缓存行的容量,每个哈希桶分成8个字,一个字用于进行同步控制,6个字用于存放键/值对,另一个字用于指向其它哈希桶。
如果需要创建1024$*$1024个哈希桶,需要额外的分配8 MB的内存用于存储同步变量。

\begin{figure}[htbp]
\centering
%\subfigure[Intel]{
\subfigure[\textit{i} = 10$^3$]{\includegraphics[width=0.45\textwidth]{Scal-u10-i1000(noset)}}
\subfigure[\textit{i} = 10$^6$]{\includegraphics[width=0.45\textwidth]{Scal-u10-i1000000(noset)}}
\caption{传统细粒度锁方法和基于RTM的粗粒度锁方法之间的性能比较}
\label{fig:htm_coarse_grained}
\end{figure}

\textbf{分析}: 根据实验结果,有两点结论:
第一,在处理大规模工作集时,使用HTM构建并发哈希表的好处体现在两个方面:一是获得的性能和扩展性具有一定的竞争力;二是它能降低内存开销并达到简化并行编程的目的。
第二,当工作集的大小小于片上缓存的容量时,此时由于更加激烈的数据冲突引发频繁的事务中止影响了基于HTM的全局锁性能。

\subsection{不同的全局锁方案之间的比较}
\label{sec:htm_global}
在这一小节,将进一步评估使用不同的锁实现方式作为全局锁时,对应的缓存行哈希表之间的性能差异。
为此实现了6种锁方法,它们分别是:
(1)标准的MCS锁——没有使用任何优化的MCS锁方法;
(2)用SLR优化的基于RTM的MCS锁(slr-mcs);
(3)基于HTM的事务重试方法(HTM-retry);不使用SLR和SCM进行优化,仅参照Intel技术手册上推荐的方法对发生中止的事务进行重试,重试的次数设定为10。
(4)使用SCM优化的基于RTM的ttas锁;
(5)采用SLR和SCM两种软件辅助方法共同优化的MCS锁(slr-scm-mcs);
(6)基于乐观的SCM方案的ttas锁,线程在经过10次重试事务执行后仍无法成功提交时,线程会以事务性的方式获取锁,完成本次操作。

图~\ref{fig:htm_global}~展示了运行初始化大小为1000和1百万两个不同规模的数据集的性能曲线。
\begin{figure}[htbp]
\centering
%\subfigure[Intel]{
\subfigure[\textit{i} = 10$^3$]{\includegraphics[width=0.45\textwidth]{coarse-grained-u10-i1000(noset)}}
\subfigure[\textit{i} = 10$^6$]{\includegraphics[width=0.45\textwidth]{coarse-grained-u10-i1000000(noset)}}
\caption{使用不同全局锁方法之间的性能比较}
\label{fig:htm_global}
\end{figure}
正如预期那样,使用传统的未经优化的MCS锁方法不论是线程扩展性还是吞吐量都是几种方案中最差的。
观察图~\ref{fig:htm_global}~(a)和(b)的性能曲线发现,在处理较小规模的数据集(初始化元素个数为1000)时,
slr-mcs,slr-scm-mcs和HTM-retries在\textit{n}为24时达到吞吐量峰值,之后吞吐量随着\textit{n}的增加而减少。
而其它两种方法,scm-ttas-mcs和scm-ttas-opt对应的性能略比传统的MCS锁好一些。
在处理小规模数据时,缓存行哈希表性能在线程数量达到一定数值之后下降的原因是:这种情况下触发数据竞争的概率成倍增加。
当数据集的规模超过最后一级缓存的容量时,slr-mcs, slr-scm-mcs以及HTM-retry三种方法都展现了近线性的线程扩展性。

通过对实验数据的分析还发现基于RTM实现的全局锁方法的缓存行哈希表的性能受线程绑定方式(三种线程绑定方式见~\ref{sec:thread_pinning}~)的影响,即采用不同的线程绑定方法时存在性能上的差异。
这里采用紧凑型线程绑定方式。
图~\ref{fig:htm_pinning}~中描述了本次实验的结果。
\begin{figure}[htbp]
\centering
%\subfigure[Intel]{
\subfigure[\textit{i} = 10$^3$]{\includegraphics[width=0.45\textwidth]{coarse-grained-u10-i1000(setcpu)}}
\subfigure[\textit{i} = 10$^6$]{\includegraphics[width=0.45\textwidth]{coarse-grained-u10-i1000000(setcpu)}}
\caption{紧凑型线程绑定方案运行结果}
\label{fig:htm_pinning}
\end{figure}
通过与图~\ref{fig:htm_global}~的比较,有两点发现:
其一,slr-scm-mcs, slr-mcs和HTM-retry三种方案的受线程绑定方式的影响较小,并且比其它几种方案的性能更加稳定。
其二,scm-ttas-mcs和scm-ttas-opt在单个socket内表现出较好的扩展性,然而当线程数量超过单个socket能够提供的最大线程数量时,它们的性能下降严重。
这种现象正好印证了基于TAS的锁在多核系统上扩展性差的结论。

在~\ref{sec:htm_analysis}~中,将通过测算总的事务量、事务中止率以及发起锁请求的次数三个指标具体分析引起不同线程绑定方式之间性能差异的原因。

\textbf{分析}:由这部分的实验结果得到的结论分为三个方面:
第一,MCS和HTM-retry同样是使用全局锁,后者是基于硬件事务内存实现的方法,但未使用任何软件优化方案,两者之间的性能差异巨大,验证了硬件事务内存有助于提高多核处理器性能的论断;
第二,使用基于HTM实现的锁方法,能够在性能上有多大的提升还取决于所使用的软件优化方案;
第三,slr-scm-mcs和HTM-retry两种方案在扩展性上较其它方法更具有竞争力。


\subsection{基于HTM的细粒度锁实现与传统细粒度方法的比较}
使用基于RTM的全局锁构建并发哈希表有利于性能的提升。
还有一个令人关心的问题,如果使用基于HTM实现的细粒度锁构建并发哈希表会对性能提升有帮助吗?
下面将通过一组实验对比数据来回答这个问题。

参照实现粗粒度RTM锁的缓存行哈希表的设计方法,增加锁的数量,缩短临界区的长度,使用基于RTM的MCS锁实现了缓存行哈希表的细粒度锁版本,并将其与原始实现的细粒度锁版本在CHTBench测试框架上运行同等规模的数据集,对两者的性能和线程扩展性进行了比较。
结果如图~\ref{fig:htm_fine_grained}~所示。
设置两组不同的初始化数量,一组用以说明哈希表的规模小于高速缓存的容量时的性能($i = 10^3$),一组用以说明哈希表的规模远超过高速缓存容量时的性能($i = 10 ^6$)。

\begin{figure}[htbp]
\centering
%\subfigure[Intel]{
\subfigure[\textit{i} = 10$^3$]{\includegraphics[width=0.45\textwidth]{fine-grained-u10-i1000}}
\subfigure[\textit{i} = 10$^6$]{\includegraphics[width=0.45\textwidth]{fine-grained-u10-i1000000}}
\caption{使用/不使用HTM的细粒度锁方法之间的性能比较}
\label{fig:htm_fine_grained}
\end{figure}

通过比较得知,\textit{n} < 48时,两个版本间的性能差异并不明显。
这种性能差异不明显的现象说明:如果采用细粒度锁方法,使用RTM进行优化以期获得更高性能的做法意义不大。
但是当$n > 48$时,图~\ref{fig:htm_fine_grained}~(a)和(b)对应的曲线有较大的差别。
具体地,当数据量减小时(图~\ref{fig:htm_fine_grained}~(a)),使用RTM的性能要略高于不使用RTM。
在这种数据量小而并发度高的情况下,两种不同的实现方式都面临严重的数据竞争问题,但是由于RTM-MCS锁设置了Retry机制,故而在这种情况下它具有比原始的细粒度锁更好的稳定性。
而在图~\ref{fig:htm_fine_grained}~(b)中,情况恰好相反,不使用RTM的版本具有更好的线程扩展性。


\textbf{分析}:经过实验比较,得到的结论是:
在传统的细粒度锁方法能够提供良好的线程扩展性和性能的前提之下,使用RTM进行优化既达不到简化并发控制和降低内存开销的目的,又对整体吞吐量的提升没有助益。

\subsection{影响HTM性能的因素分析}
\label{sec:htm_analysis}

图~\ref{fig:htm_global}~为五种不同的基于RTM的同步方案的性能曲线。
为了探究造成它们性能上的差异的原因,借助Intel的性能计数监视器(PMU)收集了一些微观的运行时指标。
在接下来的内容中只挑选最能说明观察到的实验现象的指标予以说明。
表~\ref{tab:htm_abort_rate}~中列举的数据对应的参数为$n = $32,$i$分别为1百万/1000,$u = $10\%。
表中第二列(c$_2$)数据表示执行过程中请求锁的次数,第三列(c$_3$)为发起的总的事务量,第4列(c$_4$)表示被中止的事务的数量,最后一列为中止率,通过c$_4$/c$_3$计算得到。
事务中止率\textit{r}由中止的事务量除以执行期间发起的总的事务量计算得到,总的事务量为中止的事务量与提交的事务量之和。
从表~\ref{tab:htm_abort_rate}~的数据表明不同的软件优化方案对应的中止的事务量差别不大,这说明中止的事务量并不是最主要的性能瓶颈,而执行过程中申请锁的次数以及发起的总的事务量对性能有直接的影响。
申请锁的次数和提交的事务量越多,对应的吞吐量越高。
比如,slr-mcs,scm-ttas-mcs和HTM-retry对应的申请锁的次数和提交的事务量明显高于其它两种方案,图~\ref{fig:htm_global}~所示的这三种方法的性能曲线比另外两种更加理想,也正好印证了上述论断。

\begin{table}[htbp]
  \centering
  \caption{Intel PMU收集的不同的软件优化方法的运行时数据}
  \label{tab:htm_abort_rate}
  \begin{tabular}{lllll}
    \toprule
                  & 申请锁的次数(百万) & 总的事务量(百万) & 中止的事务量 & 中止率(\%) \\
    \midrule
      HTM-retry   & 170/160           & 290/360         & 53/67      & 18.3/18.5 \\
      slr-mcs     & 160/60            & 290/360         & 52/67      & 18.6/18.6 \\
      slr-scm-mcs & 160/160           & 300/350         & 52/66      & 17.8/19.0 \\
      scm-ttas-mcs& 48/12             & 110/130         & 51/62      & 48.6/48.4 \\
      scm-ttas-opt& 110/48            & 220/180         & 53/65      & 23.0/35.4 \\
    \bottomrule
  \end{tabular}
\end{table}

另外一种现象是吞吐量随着中止率的升高而下降。
通过运行不同的参数组合来探究中止率、创建的线程的数量、更新比重以及哈希表初始化元素个数几者之间的关系。
表~\ref{tab:abort_rate_thread}~用以说明线程数与中止率之间的关系。
表~\ref{tab:abort_rate_thread}~的第二列和第三列分别表示发起的事务总量和被中止的事务量。
不论是发起的事务总量还是被中止的事务量都随着线程数量的增加而增加,然而被中止的事务量的增长速率远低于发起的事务量的增长速率。
换言之,随着越来越多的线程被创建用来参与运算,被提交的事务量的增长速度远大于被中止的事务量的增长速度。
这个趋势与图~\ref{fig:htm_fine_grained}~中的线程扩展性曲线相符合。

\begin{table}[htbp]
  \centering
  \caption{slr-scm-mcs方案的中止率随线程变化情况}
  \label{tab:abort_rate_thread}
  \begin{tabular}{llll}
    \toprule
      \textit{n}  & 总的事务量(百万) & 中止的事务量 & 中止率(\%) \\
    \midrule
      2   & 156/156 & 70/66 & 43.6/42.3  \\
      8   & 228/239 & 72/59 & 31.6/21.5 \\
      16  & 303/298 & 73/64 & 24.1/19.5  \\
      32  & 382/349 & 80/67 & 20.9/19.2  \\
      40  & 399/299 & 77/74 & 19.3/24.8 \\
      48  & 495/310 & 87/78 & 17.6/25.2  \\
      64  & 522/316 & 90/89 & 17.2/26.0 \\
    \bottomrule
  \end{tabular}
\end{table}

表~\ref{tab:htm_update}~中的数据用以说明中止率随数据集中更新比重的变化情况。
线程数为32,表的第2到第4列表示的内容与表~\ref{tab:htm_abort_rate}~相同,每一列表中记录了两组数据,分别代表初始化值为1百万和1000时的测试结果。
这一次,运行的测试集中更新比重分别为0,10\%和80\%。
处理纯读数据集时,对应的中止率是最低的。
这是因为在纯读的场景下,没有写入内存的请求,这样发生数据冲突的概率很小,从而引起的事务中止也相对较少。
随着数据集中更新比重的增加,中止率会上升,性能自然受到影响而下降。

\begin{table}[htbp]
  \centering
  \caption{slr-scm-mcs方案的中止率随更新比重变化情况}
  \label{tab:htm_update}
  \begin{tabular}{llll}
    \toprule
      \textit{n}  & 总的事务量(百万) & 中止的事务量 & 中止率(\%) \\
    \midrule
      0   &  346/554 & 73/60 & 21.1/10.8  \\
      10  &  320/352 & 70/66 & 21.9/18.8 \\
      80  &  282/273 & 74/70 & 26.4/25.6  \\
    \bottomrule
  \end{tabular}
\end{table}

\section{本章小结}
本章首先介绍了事务内存的起源、概念,并对Intel TSX提供的两种事务同步扩展指令集——HLE和RTM——的使用方法和特性进行了描述。
提出了针对Intel TSX的两种软件优化方法:SLR和SCM。

然后,从对哈希表顺序执行五次操作的实例入手,由浅入深的探讨了当前主流的三种多线程同步编程模式:粗粒度锁,细粒度锁和硬件事务内存,在构建并发哈希表上的优势和劣势。
粗粒度锁方法实现简单,易于理解,且容易保障正确性,但是使用这种方法获得的性能不理想。
细粒度锁方法能够发挥出多核系统的性能优势,并且有很好的线程扩展性,但是细粒度锁增加了设计并发哈希表的难度,并且需要花费大量精力在保证正确性上。
事务内存继承前两者的优势于一身,实现简单,易于验证正确性,又能够获得接近甚至超越细粒度锁方案的性能,是理想的并发编程范式。

接着,介绍了与本章内容密切相关的MCS锁的原理。参考基于细粒度锁的缓存行哈希表的设计方法,结合SLR和SCM两种软件辅助方法实现了基于硬件事务内存的MCS锁,并使用这种锁实现了基于RTM的缓存行哈希表,并就二者的性能进行了比较,证明了基于RTM的缓存行哈希表能够保证良好的线程扩展性和性能,并且在数据集规模较大的场景中,性能比使用细粒度锁方法时更加优秀。

实现了6种不同的粗粒度并发哈希,用于进一步的探究使用哪一种方法对基于HTM的并发哈希表的优化效果更好。
并结合实验结果,论证了在细粒度锁方案已经能够保证良好的性能和线程扩展性的前提下,使用细粒度的基于HTM的锁对于性能的提升没有意义。

最后,对影响硬件事务内存性能发挥的因素进行了分析。结合在测试过程中统计的发起的事务总量、被中止的事务量以及中止率等数据,对实验评估中的一些现象进行解释说明。

对基于硬件事务内存的缓存行哈希表的评估结果表明,本章提出的方法达到了预期目标。
对于并发哈希表的设计具有重要的意义。
