\chapter{并发哈希表的相关研究}

% 引子

\section{哈希表概述}

\subsection{相关概念}

%查询数据结构的相关概念
\textcolor{red}{查询数据结构}

%哈希表的基本概念
\textcolor{red}{哈希表}
哈希表、树、链表等都属于搜索数据结构。
搜索数据结构由元素集合以及访问和操作这些元素的接口构成。
如果搜索数据结构能够被多个处理器共享,我们则称该数据结构为\textbf{并发搜索数据结构}(CSDS)。
哈希表(hash table),又名散列表,是一种应用广泛的搜索数据结构,它通过键值对(key-value)实现对关联数据的高效存取。
键值对之间的映射关系称为\textbf{哈希函数}。
一般的哈希表都提供了\textit{add(),remove()}和\textit{find()}三种操作的接口。
哈希表的操作分为\textbf{读}操作和\textbf{写}操作,其中读操作指哈希表的查询操作,写操作包括在哈希表中插入和删除元素。
哈希表和树型数据结构相比的最大的优势是哈希表的查询复杂度可以到常数级。
存放值的存储空间称为\textbf{哈希桶}(bucket)或者\textbf{哈希槽}(slot)。
哈希表中存放的元素的数量与哈希桶数量的比值称为\textbf{负载因子}(load factor).
哈希表使用哈希函数计算得到一个索引值,该索引值表明键对应的值在桶数组中的位置。
关于哈希函数,有一个最理想的原则:将每一个key映射到单独的哈希桶内。
但是,当数据集规模很大时,能够完美的践行上述原则的哈希函数并不存在。
因此在实际的映射过程中往往会出现多个key对应相同的索引值,这时称为发生了\textbf{碰撞}(collision).
既然碰撞无法避免,那么我们能做的就是在设计哈希表的时候尽量的选择好的哈希函数。
一个好哈希函数的基本需求是输出的哈希值比较均匀。
这样可以使发生碰撞的概率最小化,同时使得各个bucket中碰撞的条目比较平均。
有国外的研究人员对已有的哈希函数做过比较~\cite{Josh2012},结论是MurmurHash3~\cite{Murmurhash}~和CityHash~\cite{cityhash}~是迄今为止最出色的哈希函数。

\subsection{哈希函数}

\subsection{哈希冲突处理}

\textcolor{red}{根据相关文献再完善}
处理碰撞的方法大致可以分为两类:一类是\textbf{开放寻址法}(open-addressing);一类是\textbf{开链法}(separate chaining)。

开链法,每一个哈希桶都是独立的,所有经过哈希之后具有相同索引值的元素都放在同一个哈希桶中,这些元素通过列表进行管理。
所以,使用开链法的哈希表能够存储的元素的个数大于哈希桶的数目,也就是说它的负载因子可以大于1.
对哈希表执行操作的时间等于找到相应的哈希桶的时间加上对列表进行操作的时间。
虽说使用开链法的哈希表的负载因子可以大于1,并不意味着链入同一个哈希桶中的元素的个数可以是无限制的,如果某一个位置冲突过多的话,插入的时间复杂度将退化为O(N)。因此,每个哈希桶内元素的个数应在3个以内。

开放寻址法,所有元素都存放在哈希桶数组内,当需要在哈希表中插入新元素时,将对哈希桶进行扫描,从被直接映射到的哈希桶开始,按照某种探测序列进行扫描,知道找到空闲的哈希桶为止。
当需要查找某个元素时,需要以同样的探测序列进行查找,直到找到所需的元素,或者最终发现元素不在表中为止。
常用于开放寻址法的探测序列有线性探测、二次探测以及双重哈希。使用开放寻址法设计的哈希表的缺点是它存储的元素的数量不能超过哈希桶的数量。
实际上,即便是再好的哈希函数,在负载因子大于0.7之后性能会急剧下降。

两种方法各有优劣,开放寻址在解决当前冲突的情况下可能会导致新的冲突,而开链不会产生这种问题。另一方面开链的局部性较之开放寻址法要差,在程序运行过程中可能引起操作系统的缺页中断,从而导致系统颠簸。

哈希表被广泛实现系统层和应用层软件,被集成到编程语言如Java,Python等,还可以用来实现关联数组,数据库索引,缓存,集合等。
哈希表的高效性使其具有重要的研究价值和应用价值。

\section{并发哈希表的同步机制研究}

\subsection{细粒度锁方法}

\subsection{无锁化编程}

\subsection{硬件事务内存}

\section{非一致性内存访问的内存管理相关研究}

\subsection{XX研究}
\label{sec:}

\subsection{XX技术}

\section{事务内存相关研究}

\subsection{实现事务内存的相关技术}
% 相关研究
\subsection{基于事务内存的并发数据结构}

\section{集合元素查询算法}

\subsection{布鲁姆过滤器算法}
在数据库、缓存、路由器和存储系统中通常需要使用判定一个元素是否存在在某个集合内,这种判断允许一定的误报率。
进行这种成员关系判定布鲁姆过滤器(bloom filter)是使用得最多的一种数据结构\cite{bloom1970space}。
布鲁姆过滤器因其高效的内存效率而备受关注。
布鲁姆过滤器被广泛应用于概率路由表中减少内存空间的需求\cite{yu2009buffalo};用于加速IP地址的最长前缀匹配\cite{dharmapurikar2003longest};用于提升网络状态管理和监控\cite{bonomi2006beyond,song2005fast};用于网络数据包组播转发信息的编码\cite{jokela2009lipsin},以及其他的网络应用\cite{broder2004network}.

\subsection{Cuckoo哈希算法}



\section{本章小结}


